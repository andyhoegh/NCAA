
\section{Luck}
\andyc{Lucas's section}
\lucasc{I like this Quote from Will C. I found while searching the forum for the basketball tourney. I think it fits too. }
\noindent No change. It's not about the prize, it's the element of luck involved that makes it 50\%.\\
\noindent  William Cukierski ,Kaggle Competition Admin \\

Having detailed popular methods for prediction and outlined our new methodology that accounts for specific match ups, the final component of this paper addresses the high degree of chance present in NCAA tournaments and consequently prediction for NCAA tournaments.  Specifically we address how sensitivity in a Kaggle style leader board . It's natural to think 2nd place almost won, but how close was 20th place to winning? 

% some verbiage in here... 

To study the effect of luck we take the idea that the 2nd place team almost won and we attempt to quantify `almost'. There are several ways one might consider to conduct a study of this sort, we examine alternate realities where a losing team in a game is treated equally as a winning team in the case of an overtime. We focus only on overtime games because it seems reasonable to consider the fact that these games could have gone either to the victor or the loser. Recalling that the loss function used for scoring of the waggle competition is

\begin{equation}
\sum_{i=1}^n\frac{y_ilog(p_i)+ (1-y_i)log(1-p_i)}{n},
\end{equation}

Where $y_i$ is a binary variable taking value 1 when the team wins and 0 otherwise and $0 \leq p_i \leq 1$ denotes the predicted value for the team in the $i$th game.  
We ask the question how dependent on the characteristics of this particular loss function is the ranking of individual teams? 

To answer this question we studied a couple of possible alternate scoring functions. We consider the following scoring functions: 
 
\begin{equation}\label{eqn:first_score_function}
-log(1-2|.5-p_i|)
\end{equation} 

\begin{equation}\label{eqn:second_score_function}
-log(1-|.5-p_i|)
\end{equation} 

\begin{equation}\label{eqn:third_score_function}
-log(1-|y_i-p_i|)
\end{equation} 
Another scoring function namely, $|.5-p_i|\times log(|.5-p_i|)$ might also be considered. We excluded this function because it highly discourages values near 0.5 while penalizing nothing for the value 0.5. Here, for brevity, we only write the part of the loss function multiplied by the $(1-y_i)$ term in the loss function above. Moreover, in Function \ref{eqn:third_score_function}, $y_i \in \{0,.5,1\}$, where now the $0.5$ is the realized value when teams go into overtime. Additionally, this scoring function encourages prediction values in the range $[0,0.3]$ or $[.7,1]$ which seems to be an undesirable effect for a scoring function. A plot of the different score functions for values $0\leq p_i \leq 1$ is shown in Figure \ref{fig:scoring_functions}.  

\begin{figure}[H]
\centering
\includegraphics[width=.7\textwidth]{loss_function_plot.pdf}
\caption{A plot of the various loss functions as a function of the prediction $p_i$.  }
\label{fig:scoring_functions}
\end{figure}

It is clear from looking at Figure \ref{fig:scoring_functions} that the different scoring functions treat the contestants confidence differently. Both the first and the second score function are symmetric about 0.5 which is aesthetically pleasing. Moreover, the first score function penalizes much more for confident predictions than the second. 

One might wonder, given the plethora of choices one might use to rank contestants, how much is a result of a specific function and how much hold sir we move from one scoring function to another? To examine this we made a plot of the contestant predictions scored under the various loss functions described above. The results of this plot are shown in Figure \ref{fig:score_rank_plot}. 

\lucasc{I just got a table from Will C. that has match ups of userID with teamID so I can match these scores to the team and plot by team.}

  \begin{figure}[H]
\centering
\includegraphics[width=0.7\textwidth]{prelim_rank_plot.pdf}
\caption{The scores of each contestant under the different loss functions. A smaller score is better.  }
\label{fig:score_rank_plot}
\end{figure}

Now these are the scores for various players and we know a leaderboard consists of teams. Some teams might include multiple players and some teams consist of only a single player. To compare our results directly to the leaderboard we merged the best predictions into team level data. The results can be seen in Figure \ref{fig:team_rank_plot}. 
Not surprisingly, the results seem somewhat similar, teams generally do not change there score, and therefore their ranking very much. 

 \begin{figure}[H]
\centering
\includegraphics[width=1.1\textwidth]{ranksScreenShot.png}
\caption{The scores of each teamunder the different loss functions. A smaller score is better.  }
\label{fig:team_rank_plot}
\end{figure}

\lucasc{Now the concluding paragraph for section}

While much has been made in contest forums and in books \cite{schutt2013doing} about leakage and how leakage is often exploited to win competitions it is somewhat refreshing to see that other choices play little, if any, into the ultimate results from a competition. While luck ultimately plays a role in the basketball games themselves and the results of prediction tournaments, some players are able to consistently outperform. These consistent performs are lucky, but lucky in the sense they are endowed with particular gifts that the rest of us mortals only dream of, and practice in hopes of achieving some semblance of success.  