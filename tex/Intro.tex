\section{Introduction}
\andyc{Ian \& Yuhyun's section}

Every March, millions of people take to an American tradition, filling out an NCAA tournament bracket.  While there are a multitude of ways to fill out a bracket,typical strategies include using the tournament seeds, listening to so-called experts on tv or the internet, or following one's intuition.  Recently, the rise in sports analytics and the popularity of sites such as Nate Silver's fivethirtyeight.com are increasing the visibility of data-driven methods for NCAA tournament predictions.  Additionally, unique tournament challenges such as the one hosted by Kaggle in 2014 call for winning probabilities for each potential matchup essentially requiring computational methods.  This work contains three major components: (i) a review analytic approaches to forecasting NCAA tournament games, (ii) an introduction our modeling framework designed to capture matchup effects, and (iii) a discussion the large effect of chance plays on tournament outcomes and hence, modeling competition results.


\subsection{March Madness}
The NCAA Men's Division I Basketball Championship began with the idea of the coach Harold Olsen from Ohio State University during 1939. Since most of games are played in March, ``March Madness" has been used to refer the NCAA basketball tournament and this term is owned by the NCAA as a registered trade mark.
For the first time, Henry V. Porter used ``March Madness" to describe an exhilaration of the Illinois state high school basketball tournaments in 1939.  After, a sportswriter and a referee, Jim Enright's work(\cite{enright1977march}), ``March Madness" became the term nationally famous and now it refers the NCAA basketball tournament. As the NCAA basketball tournament increases in popularity, predicting March Madness winners becomes the traditional event in March for American basketball fans. In addition, Turner Sports and CBS Sports have broadcasted all tournament games nationally since 2011. Due to a nationwide broadcast, TV advertising revenue is consistently increasing and March Madness in 2013 produced over $\$ 4$ billion in TV advertising revenue(\cite{Kantar2014}). For the NCAA, TV rights, ticket sales, and others such as merchandise sales and concession are the source of revenue. Specifically, revenue from TV rights and ticket sales in 2013 amounted to about $\$ 684$ million and $\$ 71$ million, respectively(\cite{NCAA2014}).

\subsection{Overview of Kaggle}
Kaggle was founded in 2010 by Anthony Goldbloom(www.kaggle.com). Since then, Kaggle has been the place to post competitions for people who are interested in discovering meaningful information from structured or unstructured big data and are eager to share their ideas with respect to predictive modelings. Energized by the popularity of March Madness, Kaggle announced the competition, ``March Machine Learning Mania" sponsored by Intel to build predictive models for the NCAA college basketball tournament in 2014. For the purposes of this competition, participants are required to build their own predictive models by using the provided historical game data and encouraged to use external data from any other sources. This competition is different from a typical bracket competition or Warren Buffett's $\$ 1$ billion contest. Rather than filling out a bracket, participants are required to submit the probability of a team winning for all possible matchup, $68 \choose 2$.

\subsection{Tournament Prediction}
There has been a lot of research for predicting outcomes of sports games and for evaluating predictors. \cite{boulier2003predicting} shows that rankings(or seedings), power scores published in \emph{The New York Times}, and sports betting markets(i.e. Las Vegas bookmakers) are informative as good predictors, forecasts the outcomes of the NFL games using a Probit regression model, and compares the accuracy of these predictions to naive forecasts(i.e. a home team will win), opinions from sports editors, and information offered from sports betting markets. Also, Jeff Sagarin, a well-known sports statistician has been providing the ranking systems in sports to USA Today's sports since 1984. For the NCAA basketball tournament, \cite{smith1999can} shows that there is a meaningful relationship between the margin of victory and the seed by applying a simple linear regression. \cite{caudill2003predicting} considers a nonparametric model which utilizes the maximum score estimator developed by \cite{manski1977estimation} in order to maximize the number of correct predictions, applies their model into the men's NCAA basketball tournament data from 1985 to 1998, and shows their model has the better performance than a Probit regression model. \cite{west2006simple} fits an ordinal logistic regression and expectation method by using the number of wins from 2003 to 2005 as predictors and predicts the probability of winning 0 through 6 in 2006 for each team. \cite{wright2012statistical} has modeled winning outcomes in the NCAA tournament through OLS and a probit regression model with the covariates such as seed, win percentage of team's regular season record, Sagarin ranking, point per games, and etc.


