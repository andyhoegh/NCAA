\documentclass[10pt,a4paper]{article}
\usepackage[utf8]{inputenc}
\usepackage{amsmath}
\usepackage{amsfonts}
\usepackage{amssymb}
\begin{document}
\section{Popular Methods}
March Madness prediction draws the attention of sport analysts and medias: notably, Nate Silver, Ken Pomeroy and the ESPN. In this section, we will compare these three models against Kaggle entries and examine the ``luck'' factor of Kaggle competition.

Because Silver and Pomeroy published probabilities of each team advancing, our first comparison focused on first round games (round of 64).  We noticed that all three models concurred in win-loss predictions except for one game (Gonzaga vs OKST, 8 vs 9 in West). All three models have a similar pattern based on the predictions' deviation from 0.5: while most predictions either suggest a easy win (prediction close to 1) or a close game (prediction close to 0.5), predictions around 0.75 are relatively rare in all three models.

In addition, Silver's prediction adopted a more aggressive style than the other two. His prediction, on average, has the highest deviation from 50-50 prediction and Pomeroy has the lowest. But this difference seems to be irrelevant to their scores. For the first around games, Pomeroy has the best score at 0.4632, followed by Silver 0.4664 and ESPN 0.4709. These scores ranked 38th, 45th and 64th respectively, among 433 Kaggle entries. 

We also computed a full-scale, team-to-team prediction based on the marginal advancement probabilities and found both the score and ranking of three models have dropped markedly from first round: ESPN (0.5795, 88th), Silver(0.5988, 123rd) and Pomeroy (0.6278, 174th). 

Another interesting finding in this comparison is the ``luck'' factor. What if a close game finished differently? To that end, we flipped the five overtime games in the first round and noticed a completely reverse of the ranking of popular models: ESPN(0.4631, 11th), Silver(0.5025,117th) and Pomeroy (0.5027,120th). This phenomenon is observed in other Kaggle entries as well. A 0.7169 Kendall's Tau is measured between the rankings before and after flipping. 

To sum up, popular models exhibit similarities in many aspects,  but none has shown clear advantages over most Kaggle entries. 
 

\end{document}
