\section{Introduction}
 Every March, millions of people take to an American tradition, filling out an NCAA tournament bracket.  Typical strategies include following one's intuition or listening to so-called experts on tv or the internet, although the rise in sports analytics and the popularity of sites such as Nate Silver's fivethirtyeight.com are increasing the use of data-driven methods.  With this work we review analytic approaches to forecasting NCAA tournament games, introduce our modeling framework designed to capture matchup effects, and discuss the effect of chance on tournament outcomes.  
\subsection{Common Prediction Methods}
Many common methods for predicting outcomes assess overall team strength using ranking based metrics.  In addition to quantify overall team strength, our methodology provides a data-driven approach to assess team by team match ups. 
\subsection{Fan Competitions}  While most competitions are concerned with filling out a traditional bracket, we also consider the recent Kaggle competition that required pairwise winning probabilities for each potential matchup.
\subsection{LitReview}
Literature Review: What people have done for predicting tournaments before (both bball and other sports are fine), March madness in general (how many people watch, how much revenue is it worth). Predicting human performance in sports (it's hard), history of Kaggle.
