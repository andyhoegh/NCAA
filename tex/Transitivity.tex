\section{Inference on Transitivity}
\andyc{This may get cut}
The transitive property states if $A>B$ and $B>C$ then $A>C$.   In terms of basketball consider:
\begin{eqnarray}
P_{A,B} > 0.5 \quad \& \quad P_{B,C} > 0.5 \rightarrow P_{A,C} > 0.5
\label{eq:trans}
\end{eqnarray}
, where $P_{I,J}$ is the probability of team I defeating team J.  Then Equation \ref{eq:trans} can be considered a transitive property on basketball match ups.  That is if team A is expected to beat team B and team B is expected to beat team C, then team A should also defeat team C.  Any sort of rank based approaches would assume this transitive ordering, home court effects non-withstanding.  Note these are probabilities not true outcomes, due to the parity in basketball inferior teams can and often do defeat stronger teams.  Nevertheless, our modeling approach can determine if the strengths of a given team present difficulties for a specific team resulting in the transitive property not necessarily holding.
