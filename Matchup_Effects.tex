\section{Model: Nearest-Neighbor Matchup Effects}
Sports broadcasters often make claims similar to \emph{Team X is a tough matchup for team Y due to their ... }.  There are two ways to consider this statement: (1) the overall team strength of Team X will be problematic for Team Y or (2) Team X has certain tendencies above and beyond their team strength that will pose difficulties for Team Y.  For the first case, models specified by Equation \ref{eq:generic} will account for differences in team strength.  However, for the second case a different approach is needed to analytically quantify characteristics that pose difficulties for a given team.  We introduce the Nearest-Neighbor Matchup Effect which captures characteristics of specific match ups.  For instance in hindsight, a glance inside the crystal ball would have revealed that Duke might struggle against Mercer.  It is not clear if this was the ten percent of instances that a 14 seed defeats a 3 seed, \andyc{check this historical frequency} or whether an uncharacteristically difficult matchup to the Mercer's characteristics for third seeded Duke.  Perhaps an astute observer - or clustering algorithm - would have recognized that Mercer was a team with similar characteristics to Clemson and Wake Forest, two teams Duke struggled against during the regular season. Then the matchup effect would have revised to initial probability of Duke winning to account for the fact that Mercer has similar characteristics to Clemson and Wake Forest resulting in a winning probability of 72 percent compared to the initial 90 percent estimate.  Computing matchup effects is a three step procedure: (1)  the typical model as in Equation~\ref{eq:generic} is fit, (2) for each matchup, past opponents most similar to the current matchup are identified, and (3) an adjustment is introduced that accounts for past performance against similar teams. 
\subsection{Relative Strength Models}
The general form for the relative strength models follows below:
\begin{eqnarray}
Y_{ij} = X_{ij} \beta + \epsilon
\label{eq:ME}
\end{eqnarray}
where $X_{ij}$ coresponds to the difference in predictors for teams $i$ and $j.$  Most commonly $X_{ij}$ will consist of differences in rankings and seeds between the two teams. 

\subsection{Choosing Neighbors}
There are a multitude of ways to select the neighbors.  In particular one needs to consider what variables to consider for selecting neighbors, how should those variables be weighted if at all, and how many neighbors should be selected.  We consider a large set of team and player level data from which a k-nearest neighbor approach is calculated.

Typically this procedure would be done analytically, however, user input can also be solicited.  For instance, suppose as in the earlier example that a user decided Mercer was similar to Wake Forest and Clemson.  In this case, Bayesian Visual Analytics (BAVA) provides a principled routine for visualizing teams and specify similarities.
\subsection{Matchup Adjustment}
The idea of the matchup adjustment is to quantify how much a team underperformed (or over performed) relative the expected level for a subset of teams similar to the current opponent.  So if team $i$ was two points better than expected against teams similar to $j$, then it would be reasonable to assume that team $i$ would perform better against team $j$ as well.  Then the predictive distribution for a matchup between team $i$ and team $j$ now becomes
\begin{eqnarray}
p(Y_{ij}|X_{ij}, \beta,\sigma^2,\mathcal{N}_i^k(j),\mathcal{N}_j^k(i), \rho) \sim N(X_{ij} \beta + \rho(\mathcal{N}_i^k(j) -\mathcal{N}_j^k(i)), \sigma^2),
\label{eq:ME}
\end{eqnarray}
where $\mathcal{N}_j^k(i)$ is the average residual for the team $i's$ $k$ past opponents most similar to team $j$ and $\rho$ is a tuning parameter $\in [0,1]$ that controls the amount of information passed from similar neighbors. 
\subsection{Tuning $\rho$}
The natural support of $\rho$ would be between zero and one.  The interpretation of the extreme points is rather intuitive - with $\rho = 0$ Equation~\ref{eq:ME} reverts to Equation~\ref{eq:generic} and with $\rho = 1$ the entire residual for similar teams is retained.
