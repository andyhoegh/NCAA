\section{Data}
The key component to any successful analytic approach is quality data.  Luckily for those predicting NCAA basketball games, a plethora of data is available.  Analysts can easily obtain some components such as wins, losses, NCAA tournament seedings as well as many common rating systems.  Others such as strength of schedule, free throw percentage, or team tempo will require more sophisticated scraping algorithms and computations or alternatively a paid subscription a to site containing the data.  Given the volume of data and the multitude of ways it can be aggregated and used, the question is what aspects are high quality predictors of tournament outcomes?  We break the data into two segments: influential factors and rankings.  First we provide a description of influential factors for basketball prediction.  Then we display an overview of common ratings metrics which incorporate many of the influential factors.  In our experience, unsurprisingly, these rating systems perform quite well.

\subsection{Influential factors} 
There are many influential factors for predicting college basketball games and NCAA tournament game in particular.  We start with game level data that has been aggregated to reflect team characteristics.  A few of these are self explanatory, such as winning percentage and point differential.  Winning percentage and point differential are largely a function of opponents played, but still provide a nice description of overall team strength.  Another important factor is home court advantage.  This is consistently shown to be worth about 4 points \cite{harville1994}.  It is also useful to have a metric to assess the strength of competition.

Many other traditional aggregated stats such as points scored, rebounds, turnovers would appear to be useful, but are quite tempo dependent.  For instance, a team with a very quick pace will tend to have more rebounds than one that plays more slowly.  This doesn't mean that the quicker pace team is a better rebounding team, just that a larger number of are rebounds available.  A simple adjustment is to use offensive and defensive rebound percentage, that is, the percent of available offensive and defensive rebounds a team collects.  Similar concepts are used to create turnover percentage, effective field goal percentage, ... \andyc{Elaborate a little here}. (IAN: This last sentence is in the passive voice. It can't be activized without a subject, so either 'We use similar concepts' or probably some citation of someone who uses them. This isn't a big deal since this isn't exactly a controversial point, but this is weasel wording.)

\subsection{Common Ratings Components} The purpose of using the influential factors is to construct a metric for overall team strength.  An alternative possibility is to use one or more of the preexisting rating systems.  For instance, Ken Massey's website www.masseyratings.com contains nearly 70 different systems.  It is no surprise that the best of these rankings work quite well.  We have found it difficult to make major improvements over these ranking systems.  While many of the algorithms are proprietary, we provide an overview of the main points for a selection of the notable ones.  Some aspects of these equations are proprietary so complete details are not possible in every case. 
\subsection{NCAA Tournament Seeds}
Each year the NCAA selection committee selects 68 teams and places them into brackets for the NCAA basketball tournament.  The seeding system, which rates teams one through sixteen in each region - not including the teams in the play in game - is well known.  A lesser known rating is the so called `S-curve' which gives an ordinal ranking of each team in the field.  
\subsubsection{Logistic Regression - Monte Carlo}
The logistic regression Monte Carlo (LRMC) method detailed in \cite{Kvam2006}, \cite{mark2010} is a two step procedure used to produce ordinal rankings of each team.  The first step evaluates every game played during the season to compute the probability that the winning team is better than the losing factor (IAN: did you mean to say team here?) .  This step uses the score at the end of regulation (e.g. overtime results are not included) and the home team in a logistic regression setting.  The second step uses these probabilities in a Markov chain to produce the ordinal rankings.  Specifically, each team is given a state in the chain based on the head-to-head probabilities and the ordinal ranking is a result of the ordering for the steady state probabilities of the chain.

\subsubsection{RPI}
The most well known rating system may very well be the Ratings Percentage Index (RPI).  The RPI was created in 1981 as a tool for evaluating teams for admission and seeding to the NCAA basketball tournament.  The RPI calculation uses three components, winning percentage (WP), opponents winning percentage (OWP), and opponents opponents winning percentage (OOWP), with weights specified as:
\begin{eqnarray*}
RPI = WP *.25 + OWP * .50 + OOWP *.25.
\end{eqnarray*}
The RPI is often criticized for heavy reliance on winning percentage and failure to account for other indicators such as point differential that illuminate the team strength.

\subsubsection{Sagarin}
Another well known ratings system is Jeff Sagarin's computer ranking system, known simply as the Sagarin rankings.  Sagarin rankings are a staple, because of both the longevity  - they have been used since 1985 - and the quality.  The actual methodology is unknown, but the Sagarin rating is actually a composition of three separate models.  One model uses only wins or losses without regard to point differential, while the other two focus primarily on point spreads.  The Sagarin rankings are unique in that difference in the team ratings represents the expected point spread for that matchup on a neutral court.  There is also a home court advantage of 3.38 points built into Sagarin's system.

\subsubsection{Pomeroy}
The Pomeroy rankings are issued by Ken Pomeroy and largely driven by the Pythagorean Expectation a formula developed by Bill James for baseball prediction.  The Pythogorean expectation is
\begin{eqnarray}
E[P(Win)] = \frac{\text{points scored}^c}{\text{points scored}^c + \text{points allowed}^c},
\end{eqnarray}
where $c$ is a constant, Pomeroy uses 10.25.  Rather than using actual points scored, Pomeroy uses adjusted (IAN: offensive?) and defensive efficiencies as inputs to the Pythogorean expectation.  Pomeroy claims that this method gives an average error of 8.25 based on backtesting and states
\begin{quotation}\noindent I don't think you can't come up with a prediction method that will have an error of less than eight points. And if you can, don't tell anyone! Because that would be a really good system. That should also tell you a lot about why it's difficult to anticipate what will happen in a single contest between teams. It's also a good illustration of the large role randomness in any single game. So even if you know it all, you can't possibly know it ALL.
\end{quotation}
This quote illustrates the difficulties inherent in basketball prediction.  Later in the manuscript we will discuss the effect of luck in the NCAA tournament and bracket prediction contests.  This format certainly is not conducive for identifying the \emph{best} team, but it is quite entertaining.
\subsubsection{Rating Comparison}
Table \ref{tab:ranks} contains pre-tournament rankings from common sources.  The table includes the top 16 seeds in the NCAA tournament as well as the eventual national champion Connecticut, who didn't make any of these top 10 lists.
\begin{table}[h!]
\caption{Pre-tournament Ranking Comparison}
\small
\centering
\begin{tabular}{l|ccccc|c}
  \hline
  \hline
 Team & Sagarin Rank &  Pomeroy Rank & RPI & LRMC Rank & Seed& Ave. Rank  \\ 
  \hline
 Arizona         & 1  &1    & 2     & 2 & 2& 1.6  \\
 Florida          & 3  &3    &1      &3 & 1& 2.2\\
  Kansas         & 6  &8    & 3    & 4& 7 &5.6\\
 Virginia         & 5  &4     &8    &8 & 4 &5.8\\
 Wichita St    & 12 &5      & 4    &5 & 3 &5.8\\
 Villanova      & 4  &7    & 5    & 9 & 5 &6\\
 Duke             & 7  &6     &9     &6 &9&7.4\\
 Louisville      & 2  &2    & 19   & 1 &13 & 7.4\\
  Creighton &  11 &   9 & 10   &7 &11& 9.6\\ 
 Wisconsin  &   9   &13   & 6   &  11 & 8 &9.4\\
 Michigan & 10 & 15& 11& 16& 6 &11.3\\
 Michigan St & 8  &10   & 18 & 12& 14&12.4\\
 UCLA & 15& 18& 14&10 &15 &14.4\\
 Iowa St &13 &23  &7 &19 &12 &14.8 \\
 Syracuse &19 &14  &16 &24 &10 &16.6 \\
 San Diego St&22 &21  &15 &25 &16 &19.8 \\
  \hline
  Connecticut & 24& 26& 22&26& 26&24.8\\
  \hline
   \hline
\end{tabular}
\label{tab:ranks}
\end{table}
There are some similarities but each of the ranking system has its own flavor.