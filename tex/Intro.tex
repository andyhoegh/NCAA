\section{Introduction}
Every March, millions of people take to an American tradition, filling out an NCAA tournament bracket. There are numerous ways to fill out a bracket. Typical strategies include using the tournament seeds, listening to so-called experts on tv or the Internet, or following one's intuition. The rise in sports analytics and the popularity of sites such as Nate Silver's fivethirtyeight.com is increasing the visibility of data-driven methods for NCAA tournament predictions. Predictive tournament challenges, such as the one hosted by Kaggle in 2014, require competitors to compute winning probabilities for every potential tournament match up, necessitating computational methods. This work stems from the work our team did for this competition and contains three major components: (i) a review of analytic approaches to forecasting NCAA tournament games, (ii) an introduction to our modeling framework designed to capture match up effects, and (iii) a discussion of the large effect that chance plays on tournament outcomes and modeling competition results.


\subsection{March Madness}
The term ``March Madness" was used for the first time by Henry V. Porter to describe the exhilaration of the Illinois state high school basketball tournaments in 1939.  Afterwards, sportswriter and referee Jim Enright's work (\cite{enright1977march}) propelled the term ``March Madness" to international fame referring to the NCAA basketball tournament. As the NCAA  tournament increased in popularity, predicting March Madness winners became a tradition for American basketball fans. Due to a nationwide broadcast, TV advertising revenue is consistently increasing and March Madness in 2013 produced over $\$ 1.15$ billion in TV advertising revenue(\cite{Kantar2014}). For the NCAA, TV rights, ticket sales, and others such as merchandise sales and concession are the source of revenue. Specifically, revenue from TV rights and ticket sales in 2013 amounted to about $\$ 684$ million and $\$ 71$ million, respectively(\cite{NCAA2014}).

\subsection{Overview of Kaggle}
Kaggle was founded in 2010 by Anthony Goldbloom (www.kaggle.com). In April 2010, Kaggle kicked off its first competition, ``Forecast Eurovision Voting”, for predicting vote results of the 2010 Eurovision Song Contest final (\cite{kagglefirst}). Since its first competition, Kaggle has become the place to go for predictive modeling competitions. Now, Kaggle networks with over 1 million data scientists and has become the largest online community for data scientists (\cite{kaggleuser}). Energized by the popularity of March Madness, Kaggle announced the competition ``March Machine Learning Mania" sponsored by Intel to build predictive models for the NCAA college basketball tournament in 2014. For the purposes of this competition, participants are required to build their own predictive models by using provided historical game data and are encouraged to use external data from any other sources. This competition is different from a typical bracket competition or Warren Buffett's $\$ 1$ billion contest. Rather than filling out a bracket, participants are required to submit the probability of a team winning for all possible $68 \choose 2$ match ups.

\subsection{Tournament Prediction Review}
There has been a lot of research for predicting outcomes of sports games and for evaluating predictors. \cite{boulier2003predicting} shows that rankings(or seedings), power scores published in \emph{The New York Times}, and sports betting markets(i.e. Las Vegas bookmakers) are informative as good predictors. Jeff Sagarin, a well-known sports statistician, has provided the ranking systems used in USA Today's sports since 1984. For the NCAA basketball tournament, \cite{smith1999can} shows that there is a meaningful relationship between the margin of victory and the seed by applying a simple linear regression. \cite{caudill2003predicting} considers a nonparametric model which utilizes the maximum score estimator developed by \cite{manski1977estimation} in order to maximize the number of correct predictions. They applied their model to the men's NCAA basketball tournament data from 1985 to 1998, and showed that their model has better performance than a probit regression model. \cite{west2006simple} fits an ordinal logistic regression and expectation method by using the number of wins from 2003 to 2005 as predictors and predicts the probability of winning 0 through 6 in 2006 for each team. \cite{wright2012statistical} has modeled winning outcomes in the NCAA tournament through OLS and a probit regression model with the covariates such as seed, win percentage of team's regular season record, Sagarin ranking, and point per games. \cite{rosenthal} considered multiple regression models, along with constrained versions thereof and a Monte Carlo search algorithm to predict team performance. In the end, he used a regression model including regular season win percentage, final three game win percentages, offensive and defensive efficiency ratings, strength of schedule, and out-of-conference win rates. \cite{ezekowitz2013} developed a proportional hazards model based on network analysis. He used offensive and defensive ratings, schedule strength, tournament experience, consistency (defined as the variance of a team's point spreads), and the number of tournament teams a team defeated during the regular season. \cite{Kvam2006} consider a logistic regression/Markov chain model to determine tournament team rankings. Higher steady state winning probabilities correspond to higher rankings. These probabilities are computed using a logistic regression using win percentages.


