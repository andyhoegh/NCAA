\section{Luck}
how sensitive is the leader board. It's natural to think 2nd place almost won, but how close was 20th place to winning? 

% some verbiage in here... 

To study the effect of luck take the idea that the 2nd place almost won and attempt to quantify `almost'. There are several ways one might consider to conduct a study of this sort, we examine alternate realities where a losing team in a game becomes a winning team in an alternate reality. For example, suppose in the last game of the 2014 tournament, instead of UConn winning Kentucky won instead. If we were examining player level data there would be the option to examine an alternate reality at the player level, here for example, instead of Aaron Harrison missing the three pointer he made the three pointer that would have given Kentucky the lead. Also suppose Kentucky only had 7 missed free throws instead of 11 in the final game. These are the types of `what ifs' we would examine if we made the choice to operate at the player resolution. Instead we operate at the team resolution and this makes the `what if' analysis somewhat simpler. 

At the $kth$ round of the tournament we have $2^{7-k}$ teams, and therefore in the first round $2^5 = 32$ teams are knocked out. If we started at this level of the tournament, the `what if' scenarios we would need to entertain would be $2^64 \approx e^{19.266}\approx 232872115$ different scenarios just at the first round, this is clearly prohibitive. Moreover, at round $k$, for $1\geq k \leq 6$ there will be $\sum_{j=k}^62^{6-j}$ more teams to be knocked out of the tournament. This means uf we do a what if at the regional we have 8 possibilities and at the elite 8 a what if analysis gives us 64 possibilities of championship winners/histories. \textbf{Question for team: Are these calculations correct? I think I'm off by a factor of $2^k$ somewhere}


(Lucas: sensitivity study)
