\section{Introduction}
\andyc{Ian \& Yuhyun's section}

 Every March, millions of people take to an American tradition, filling out an NCAA tournament bracket.  While there are a multitude of ways to fill out a bracket,typical strategies include using the tournament seeds, listening to so-called experts on tv or the internet, or following one's intuition.  Recently, the rise in sports analytics and the popularity of sites such as Nate Silver's fivethirtyeight.com are increasing the visibility of data-driven methods for NCAA tournament predictions.  Additionally, unique tournament challenges such as the one hosted by Kaggle in 2014 call for winning probabilities for each potential matchup essentially requiring computational methods.  This work contains three major components: (i) a review analytic approaches to forecasting NCAA tournament games, (ii) an introduction our modeling framework designed to capture matchup effects, and (iii) a discussion the large effect of chance plays on tournament outcomes and hence, modeling competition results.

\subsection{March Madness}
 \textcolor{red}{Paragraph or section on March madness in general (how many people watch, how much revenue is generated, ect...)}\\
 The NCAA Men's Division I Basketball Championship began with the idea of the coach Harold Olsen from Ohio State University during 1939. Since most of games are played in March, "March Madness" has been used to refer the NCAA basketball tournament and this term is owned by the NCAA as a registered trade mark.
 For the first time, Henry V. Porter used "March Madness" to describe an exhilaration of the Illinois state high school basketball tournaments and published his essay, "March Madness" in 1939.  After, a sportswriter and a referee, Jim Enright wrote his book ,"March Madness: The Story of High School Basketball in Illinois." to cover the boys basketball tournament's history in 1977, "March Madness" became the term nationally famous.  Even though, "March Madness"  originated in the boys basketball tournament in Illinois, today's "March Madness" is the vocabulary referring to the NCAA basketball tournament.
As the NCAA basketball tournament is getting popular, filling a bracket to predict which team will win becomes the traditional event in March for American basketball fans. In addition, Turner Sports and CBS Sports have broadcasted all tournament games nationally since 2011. Due to a nationwide broadcast, an advertisement revenue each year is increasing and March Madness in 2013 produced an advertisement revenue,$\$ 1,152$ million, the highest recorded ever(Source:Kantar Media).

\subsection{Overview of Kaggle}
\textcolor{red}{History, ect...}    
Kaggle is founded in 2010 by Anthony Goldbloom. Since then, Kaggle has been the place to post competitions for people who are interested in discovering meaningful information from structured or unstructured big data and are eager to share their ideas with respect to predictive modelings. Energized by the popularity of March Madness, Kaggle announced a competition, "March Machine Learning Mania" to build the predictive model for NCAA college basketball tournament. Participants were required to build their own predictive models by using the provided historical game data and the external data from any other sources, and to submit their winning percentages to the all possible matchup.

\subsection{Tournament Prediction}
\andyc{add these references to bib file and use cite command}

There have been a lot of researches to predict outcomes of other sports games as well as the NCAA men's basketball tournament and to evaluate the predictors. Boulier et al.(1999) showed that the rankings, power scores published in \emph{The New York Times}, and the betting market are informative as the good predictors, forecasted the outcomes of the NFL games by the Probit regression, and compared the accuracy of predictions with other models, such as a naive model and opinions from sports editors. Also, Jeff Sagarin, a well-known sports statistician has been providing the ranking systems in sports to USA Today's sports since 1984. For the NCAA basketball tournament, Smith and Schwertman(1999) showed that there is a meaningful relationship between the margin of victory and the seed by applying the simple linear regression. Caudill(2003) considered a semiparametric model which uses the maximum score estimator developed by Manski to maximize the number of correct predictions, applied their model into the men's NCAA basketball tournament data from 1985 to 1998, and showed their model had the better performance than the probit models. West(2006) fitted an ordinal logistic regression and expectation method by using the number of wins from 2003 to 2005 as predictors and predicted the probability of winning 0 through 6 in 2006 for each team. Wright (2012) has modeled this NCAA tournament by using the OLS regression model and the Probit regression model with the variables such as seed, win percentage of team's regular season record, Sagarin ranking, point per games, and etc. 
 

Methods for predicting tournaments, (ball and other sports), difficulties of prediction human performance in sports (its hard)
