\documentclass[10pt,a4paper]{article}
\usepackage[utf8]{inputenc}
\usepackage{amsmath}
\usepackage{amsfonts}
\usepackage{amssymb}
\begin{document}
\section{Popular Methods}
March Madness prediction draws the attention of sport analysts and medias: notably, Nate Silver, Ken Pomeroy and the ESPN. In this section, we will score these three models according to Kaggle's loss function. Also since Nate Silver and Ken Pomeroy only published probabilities of each team advancing to a certain round, our first comparison is based on first round games (round of 64) only. 
  
We first notice that all three models concurred in win-loss predictions for all games  except one (Gonzaga vs OKST, 8 vs 9 in West). All three models have a similar pattern based on the predictions' deviation from 0.5. While most predictions suggest a easy win (prediction close to 1) or a close game (prediction close to 0.5), predictions around 0.75 are relatively rare. This is may because the first around game are relatively easy prediction.

Silver's predictions also adapted a more aggressive style than the other two. His predictions, on average, have the highest 0.2818  deviation from 50-50 prediction, Pomeroy averages the lowest, 0.2388. But this difference seems to be independent of the result through, for the first around, Pomeroy has the best score of 0.4632, Silver scored 0.4664 and 0.4709 for ESPN. 

We have also examined the luck factor with popular models. After flipping the results of five overtime games, we noticed a completely reverse of the rankings of scores with the ESPN model ranks first (0.4631) and Pomeroy ranks third (0.5027), Silver remains second (0.5025).

The scores look competitive but they are based on the easiest prediction in the tournament. To obtain a fuller comparison, we compute team-to-team prediction based marginal advancement probabilities and  the overall scores are ESPN (0.5795), Silver(0.5988) and Pomeroy (0.6278). The overall scores are substantially lower than the first round's which is a common observation for most Kaggle entries.    

To sum up, popular models exhibit similarities in many aspect which can be useful for future Kaggle competitors. But result-wise, none have shown advantages over the majority of Kaggle entries. 
 

\end{document}
