\section{Discussion}
As shown in the the section on luck, chance coupled with a relatively small sample size makes this NCAA prediction a difficult task.  Nevertheless, we provide some sound methods and techniques for filling out a bracket or making game-by-game predictions.  Our matchup effects model is an intuitive means for adjusting probabilities based on matchup specific characteristics.  While there undoubtably is a more principled way to carry out this exercise, and given the time constraints, we found this to be effective.

Other considerations would be to use injury data for players.  While many competitions will not allow updates once the tournament proceeds, knowing the status of Kansas's Joel Embiid would have shifted probabilities for Kansas or similarly the effect of Iowa State's Georges Niang injury in the sweet sixteen matchup with Connecticut.

Something else that may be beneficial would be to use the score with two minutes to go, for example, to quantify how a game was.  As any fan can attest, the strategy of fouling can lead to dramatically different scores as can garbage time minutes by reserves. 

An untapped section would be looking at player level data and potentially player level match ups. 
