\section{Data}
The key component to any successful analytic approach is quality data.  For those predicting NCAA basketball games, a plethora of data is available. While tidbits like \emph{Duke is undefeated in the first round of the tournament on even years when playing a team from the state of Georgia} might be entertaining for television viewers they are nonsensical for prediction.  In fact, this particular one would no longer hold after Mercer's upset of Duke this year. Nevertheless, some components with predictive power such as wins, losses, NCAA tournament seedings as well as many common rating systems can easily be obtained. Other factors such as strength of schedule, free throw percentage, and team tempo require more work for collection and analysis. Given the volume of data and the multitude of ways it can be aggregated and used, the question is what aspects are high quality predictors of tournament outcomes? We break the data into two segments: influential factors and rankings. First we provide a description of influential factors for basketball prediction. Then an overview of common ratings metrics which incorporate many of the influential factors will be displayed. In our experience, unsurprisingly, these rating systems perform quite well.

\subsection{Influential factors} 
There are many influential factors for predicting college basketball games and NCAA tournament game in particular. We first start with game level data that has been aggregated to reflect team characteristics. A few of these are self explanatory, such as winning percentage and average point differential. Winning percentage and average point differential are largely a function of opponents played, but still provides a nice description of overall team strength. Hence, some metric assessing and controlling for the strength of competition is also prudent. Other useful factors include team height, free throw percentage, percent of points scored on three pointers. Another important factor is home court advantage which is consistently shown to be worth about 4 points (\cite{harville1994}).  

Many other traditional, aggregated stats such as points scored, rebounds, and turnovers would appear to be useful, but are  tempo dependent. For instance, a team with a very quick pace will tend to have more rebounds than one that plays slower. This doesn't mean that the quicker pace team is a better rebounding team, given the larger number of rebounds available to be had. A simple adjustment would be to use offensive and defensive rebound percentage - that is the percent of available offensive and defensive rebounds a team collects. Similarly points scored and points allowed are adjusted to control for the total number of possessions a team has during a game.  Ken Pomeroy defines adjusted offensive efficiency and adjusted defensive efficiency as the number of points scored or allowed over 100 possessions (\cite{kenpom.com}.)

\subsection{Common Ratings Components} The purpose of using the influential factors is to construct a metric for overall team strength. An alternative possibility is to use one or more of the preexisting rating systems. With the large number of people working in this area, Ken Massey's website www.masseyratings.com contains nearly 70 different systems, it is no surprise that the best of these rankings work quite well and we have found it difficult to make major improvements over these ranking systems. While some of the algorithms are proprietary, we provide an overview of the main points for select algorithms.\subsubsection{NCAA Tournament Seeds}
Currently, the NCAA selection committee selects 36 teams in addition to the 32 teams conference champions and places the 68 teams into brackets for the NCAA basketball tournament. The seeding system, which rates teams one through sixteen in each region - not including the teams in the play in game - is well known. A lesser known rating is the so called `S-curve' which gives an ordinal ranking of each team in the field.  
\subsubsection{Logistic Regression / Monte Carlo}
The Logistic Regression / Monte Carlo (LRMC) method detailed in (\cite{Kvam2006} and \cite{mark2010}) is a two step procedure used to produce ordinal rankings of each team. The first step evaluates every game played during the season to compute the probability that the winning team is better than the losing team. This step uses the score at the end of regulation (e.g. overtime results are not included) and the home team in a logistic regression setting. The second step uses these probabilities in a Markov chain to produce the ordinal rankings. Specifically, each team is given a state in the chain based on the head-to-head probabilities and the ordinal ranking is a result of the ordering for the steady state probabilities of the chain.

\subsubsection{RPI}
The most well known rating system may very well be the Ratings Percentage Index (RPI). The RPI was created in 1981 as a tool for evaluating teams for admission and seeding to the NCAA basketball tournament. The RPI calculation uses three components, Winning Percentage (WP), Opponents Winning Percentage (OWP), and Opponents Opponents Winning Percentage (OOWP), with weights specified as:
\begin{eqnarray*}
RPI = \frac{WP}{4} + \frac{OWP}{2} + \frac{OOWP}{4}.
\end{eqnarray*}
The RPI is often criticized for heavy reliance on winning percentage and failure to account for other indicators such as point differential that illuminate the team strength.

\subsubsection{Sagarin}
Another well known ratings system is Jeff Sagarin's computer ranking system, known simply as the Sagarin rankings. Sagarin rankings are a staple, because of both the longevity  - they have been used since 1985 - and the quality. The exact methodology is unknown, but the Sagarin rating is actually a composition of three separate models. One model uses only wins or losses without regard to point differential, while the other two focus primarily on point spreads. A characteristic of the Sagarin rankings is that difference in the team ratings represents the expected point spread for that matchup on a neutral court. For the 2013-2013 year the home court advantage is estimated as 3.38 points (\cite{sagarin}).
\subsubsection{Pomeroy}
The Pomeroy rankings are issued by Ken Pomeroy and largely driven by the Pythagorean Expectation, a formula developed by Bill James for baseball prediction (\cite{james}).  A theoretical derivation of the Pythagorean Expectation using a Weibull distribution \cite{miller2007}. The Pythogorean expectation is
\begin{eqnarray}
E[Pr(Win)] = \frac{\text{points scored}^c}{\text{points scored}^c + \text{points allowed}^c},
\end{eqnarray}
where points scored and points allowed are season totals and $c$ is a constant, Pomeroy uses 10.25.  Rather than actual points scored Pomeroy uses adjusted and defensive efficiencies as inputs to the Pythogorean expectation, which gives an average error of 8.25 points based on backtesting.  On a related note, on his website \cite{kenpom.com2} states:
\begin{quote}
I don't think you can come up with a prediction method that will have an error of less than eight points. And if you can, don't tell anyone! Because that would be a really good system. That should also tell you a lot about why it's difficult to anticipate what will happen in a single contest between teams. It's also a good illustration of the large role randomness in any single game. So even if you know it all, you can't possibly know it ALL.
\end{quote}
This quote illustrates the difficulties inherent in basketball prediction. Later in the manuscript we discuss the effect of luck in the NCAA tournament and bracket prediction contests. This format certainly is not conducive for identifying the \emph{best} team, but it is quite entertaining. \andyc{move this quote up to introduction: section on prediction?}
\subsubsection{Rating Comparison}
The most popular ranking systems may be the Associated Press and Coaches top 25 polls, which aggregate votes by coaches or media members. Unfortunately, votes are only tallied for the top 25 teams on each ballot; hence, they are incomplete and not considered here. Table \ref{tab:ranks} contains pre-tournament rankings for the top 16 seeds in the NCAA tournament as well as the eventual national champion Connecticut, who was a 7th seed and 26$th$ in the S-curve.
\begin{table}[h!]
\caption{Pre-tournament Ranking Comparison}
\footnotesize
\centering
\begin{tabular}{l|ccccc|c}
  \hline
  \hline
 Team & Sagarin Rank &  Pomeroy Rank & RPI & LRMC Rank & Seed& Ave. Rank  \\ 
  \hline
 Arizona         & 1  &1    & 2     & 2 & 2& 1.6  \\
 Florida          & 3  &3    &1      &3 & 1& 2.2\\
  Kansas         & 6  &8    & 3    & 4& 7 &5.6\\
 Virginia         & 5  &4     &8    &8 & 4 &5.8\\
 Wichita St    & 12 &5      & 4    &5 & 3 &5.8\\
 Villanova      & 4  &7    & 5    & 9 & 5 &6\\
 Duke             & 7  &6     &9     &6 &9&7.4\\
 Louisville      & 2  &2    & 19   & 1 &13 & 7.4\\
  Creighton &  11 &   9 & 10   &7 &11& 9.6\\ 
 Wisconsin  &   9   &13   & 6   &  11 & 8 &9.4\\
 Michigan & 10 & 15& 11& 16& 6 &11.3\\
 Michigan St & 8  &10   & 18 & 12& 14&12.4\\
 UCLA & 15& 18& 14&10 &15 &14.4\\
 Iowa St &13 &23  &7 &19 &12 &14.8 \\
 Syracuse &19 &14  &16 &24 &10 &16.6 \\
 San Diego St&22 &21  &15 &25 &16 &19.8 \\
  \hline
  Connecticut & 24& 26& 22&26& 26&24.8\\
  \hline
   \hline
\end{tabular}
\label{tab:ranks}
\end{table}
There are some similarities but each of the ranking system has its own flavor.