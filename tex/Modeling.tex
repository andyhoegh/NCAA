\section{Modeling}
One interesting dilemma involves whether the outcomes should be modeled in a binary sense (win or loss) or rather should a continuous metric such as the point spread be used.  In theory, point spread provides a means for eliciting the relative strength of one team, although as any basketball fan can attest to the final score is often not indicative of how close the game was.  A common strategy is for the trailing team to foul in the closing moments of the game, which can often result in a two point deficit turning into a ten point loss.  For this reason, we scrape scores with 2 minutes left in the game \andyc{I think this can be done}.  A comparison of these three data aggregation methods is used on cross-validated data from... \andyc{hmm... I wonder if there is a way to tune/transform point spread to control for blowouts...}
\subsection{Model Specification}
The general form for a linear model for point spread follows below:
\begin{eqnarray}
Y_{ij} = X_{ij} \beta + \epsilon_{ijk}
\label{eq:generic}
\end{eqnarray}
where $X_{ij}$ corresponds to the difference in predictors for teams $i$ and $j$ and $\epsilon_{ijk} \sim N(0,\sigma^2).$ 
\subsection{Calibrating Predictions via predictive intervals} Discuss tuning the tails for predictions
