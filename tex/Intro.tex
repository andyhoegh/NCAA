\section{Introduction}
\andyc{Ian \& Yuhyun's section}

Every March, millions of people take to an American tradition, filling out an NCAA tournament bracket. While there are many ways to fill out a bracket, typical strategies include using the tournament seeds, listening to so-called experts on tv or the Internet, or following one's intuition. Recently, the rise in sports analytics and the popularity of sites such as Nate Silver's fivethirtyeight.com is increasing the visibility of data-driven methods for NCAA tournament predictions. Additionally, unique tournament challenges, such as the one hosted by Kaggle in 2014, call for predicting win probabilities for every potential match up, necessitating the use of computational methods. This work follows from the work our team put into the most recent Kaggle competition and contains three major components: (i) a review of analytic approaches to forecasting NCAA tournament games, (ii) an introduction to our modeling framework designed to capture match up effects, and (iii) a discussion of the large effect chance plays on tournament outcomes and hence modeling competition results.

\subsection{March Madness}
\textcolor{red}{Paragraph or section on March madness in general (how many people watch, how much revenue is generated, ect...)}\\
The NCAA Men's Division I Basketball Championship began with an idea from coach Harold Olsen of Ohio State University during 1939. Since most of games are played in March, he coined the term "March Madness" to refer to the NCAA basketball tournament. This term has since been copyrighted by the NCAA. Even though the term "March Madness" originated in a boys basketball tournament in Illinois, today the term refers to the NCAA basketball tournament. As the NCAA basketball tournament increases in popularity, predicting the NCAA tournament is becoming a past time for American basketball fans. In addition, Turner Sports and CBS Sports have broadcasted all tournament games nationally since 2011. Due to a nationwide broadcast, an advertisement revenue each year is increasing and March Madness in 2013 produced an advertisement revenue,$\$ 1.15$ billion, the highest recorded ever (Source:Kantar Media).

\subsection{Overview of Kaggle}
\textcolor{red}{History, ect...}    
Kaggle was founded in 2010 by Anthony Goldbloom. Since then, Kaggle has been the place to go for people who are interested in data based competitions. Energized by the popularity of March Madness, Kaggle announced a competition, "March Machine Learning Mania" to build predictive models for the NCAA college basketball tournament. Participants were required to build their own predictive models by using the provided historical game data and external data from any other sources and to submit their winning percentages for all possible matchups.

\subsection{Tournament Prediction}
Many researchers have developed methods not only to predict the outcomes the NCAA tournaments, but also to quantify the predictive relevance of their data. Boulier et al.(1999) showed that the rankings, power scores published in \emph{The New York Times}, and the betting market are good predictors. They forecasted the outcomes of NFL games using Probit regression and compared the accuracy of predictions with other models, such as a naive model and opinions from sports editors. Jeff Sagarin, a well-known sports statistician, has been providing the ranking systems in sports to USA Today's sports since 1984. For the NCAA basketball tournament, Smith and Schwertman(1996) showed that there is a meaningful relationship between the margin of victory and the seed by applying a simple linear regression. West(2006) fitted an ordinal logistic regression and expectation method by using the number of wins from 2003 to 2005 as predictors and predicted the probability of winning 0 through 6 in 2006 for each team. Wright (2012) has modeled this NCAA tournament by using the OLS regression model and the Probit regression model with the variables such as seed, win percentage of team's regular season record, Sagarin ranking, point per games, and etc. Caudill(2003) proposed a nonparametric model which uses the maximum score estimator to predict the discrete outcomes, applied their model into the men's NCAA basketball tournament, and showed their model had the better performance than the probit model.
 

Methods for predicting tournaments, (ball and other sports), difficulties of prediction human performance in sports (its hard)
