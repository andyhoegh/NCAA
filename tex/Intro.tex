\section{Introduction}
 Every March, millions of people take to an American tradition, filling out an NCAA tournament bracket.  While there are a multitude of ways to fill out a bracket,typical strategies include using the tournament seeds, listening to so-called experts on tv or the internet, or following one's intuition.  Recently, the rise in sports analytics and the popularity of sites such as Nate Silver's fivethirtyeight.com are increasing the visibility of data-driven methods for NCAA tournament predictions.  Additionally, unique tournament challenges such as the one hosted by Kaggle in 2014 call for winning probabilities for each potential matchup essentially requiring computational methods.  This work contains three major components: (i) a review analytic approaches to forecasting NCAA tournament games, (ii) an introduction our modeling framework designed to capture matchup effects, and (iii) a discussion the large effect of chance plays on tournament outcomes and hence, modeling competition results.
   
\subsection{March Madness}  Paragraph or section on March madness in general (how many people watch, how much revenue is generated, ect...)
\subsection{Overview of Kaggle} History, ect...
\subsection{Tournament Prediction}
Methods for predicting tournaments, (ball and other sports), difficulties of prediction human performance in sports (its hard)
