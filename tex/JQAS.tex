\documentclass[letterpaper,12pt]{article}
\usepackage{graphicx,amsmath} % support the \includegraphics command and options
\usepackage{color}
\usepackage{hyperref,float}
\usepackage{dgjournal} 
%\usepackage{mathptmx}
\usepackage[authoryear,comma,longnamesfirst,sectionbib]{natbib} 

\newcommand{\andyc}[1]{[{\color{red}\sc Andy says: {\tt #1}}]}
\newcommand{\scotlandc}[1]{[{\color{red}\sc Scotland says: {\tt #1}}]}
\newcommand{\lucasc}[1]{[{\color[rgb]{0,0.7,0}\sc Lucas says: {\tt #1}}]}
\newcommand{\xinranc}[1]{[{\color{red}\sc Xinran says: {\tt #1}}]}
\newcommand{\yuhyunc}[1]{[{\color{red}\sc Yuhyun says: {\tt #1}}]}
\newcommand{\ianc}[1]{[{\color{red}\sc Ian says: {\tt #1}}]}
\newcommand{\marcosc}[1]{[{\color{blue}\sc Marcos is confused and says: {\tt #1}}]}

\oddsidemargin=0.25in
\evensidemargin=0.25in
\textwidth=6in
\textheight=8.75in
\topmargin=-.5in
\footskip=0.5in

\graphicspath{{figures/}{../figures/}}

\title{Nearest-Neighbor Matchup Effects:  Accounting for team match ups in the NCAA tournament}
 \author{The Lemanski Sports Analytics Group}        
\begin{document}
%% Do NOT include any fronmatter information; including the title, author names,
%% institutes, acknowledgments and title footnotes (author information, funding
%% sources, etc.). Start the document with the first section or paragraph of
%% the article.


\originalmaketitle
\section*{Task Summary: First round of composition Due June 13\\
Final Manuscript Due July 15}

Current Assignments:
\begin{itemize}
\item Lucas - Luck
\item Xinran - Popular Methods
\item Marcos - Recommendations, Decision Theory
\item Ian/Yuhyun - Lit Review - Introduction
\item Andy : Modeling, Evaluation, Data, Discussion
\end{itemize}

\andyc{I thought it might be worthwhile to put a list of papers to incorporate as reference.  Ian and Yuhyun are likely working on something similar.}\\
Articles to incorporate (likely include solid list of references themselves):
\begin{itemize}
\item \cite{cattelan2013}: An application of dynamic Bradley-Terry model that allows team strength to vary of course of tournament.  Also has a fruitful list of references.
\item \cite{Kvam2006}: An overview of the logistic regression Markov Chain (LRMC) method for prediction.  A good read, which also contain useful citations.
\end{itemize}


\section{Introduction}
 Every March, millions of people take to an American tradition, filling out an NCAA tournament bracket.  While there are a multitude of ways to fill out a bracket,typical strategies include using the tournament seeds, listening to so-called experts on tv or the internet, or following one's intuition.  Recently, the rise in sports analytics and the popularity of sites such as Nate Silver's fivethirtyeight.com are increasing the visibility of data-driven methods for NCAA tournament predictions.  Additionally, unique tournament challenges such as the one hosted by Kaggle in 2014 call for winning probabilities for each potential matchup essentially requiring computational methods.  This work contains three major components: (i) a review analytic approaches to forecasting NCAA tournament games, (ii) an introduction our modeling framework designed to capture matchup effects, and (iii) a discussion the large effect of chance plays on tournament outcomes and hence, modeling competition results.
   
\subsection{March Madness}  Paragraph or section on March madness in general (how many people watch, how much revenue is generated, ect...)
\subsection{Overview of Kaggle} History, ect...
\subsection{Tournament Prediction}
Methods for predicting tournaments, (ball and other sports), difficulties of prediction human performance in sports (its hard)

%%%%%%%%%%%%%%%%%%%%%%%%%%%%%%%%%%%%%%%%%%%
\section{Data}
The key component to any successful analytic approach is quality data.  Luckily for those interested in predicting NCAA basketball games, a plethora of data is available.  Some components such as wins, losses, NCAA tournament seedings as well as many common rating systems can easily be obtained.  Others such as strength of schedule, free throw percentage, or team tempo will require more sophisticated scraping algorithms and computations or alternatively paid subscriptions to sites containing the data.  Given the multitude of ways the data can be aggregated and used, the question remains what aspects are high quality predictors of tournament outcomes?  First an explanation of influential factors will be provided.  Then an overview of common ratings metrics which incorporate many of the influential factors will be displayed.  In our experience, unsurprisingly, these rating systems perform quite well.

\subsection{Influential factors} 
There are many influential factors for predicting college basketball games and NCAA tournament game in particular.  A few of these are self explanatory, such as: tournament seed, winning percentage, and home court advantage.  Winning percentage is largely a function of opponents played, but still provides a nice description of overall team strength.  Home court advantage is consistently shown to be worth about 4 points \andyc{cite here}.

Many other traditional stats such as points scored, rebounds, turnovers would appear to be useful, but are quite tempo dependent.  These can be `standardized' by using adjusted metrics.  For instance, a team with a very quick pace will tend to have more rebounds than one that plays slower.  This doesn't mean that the quicker pace team is a better rebounding team, given the larger number of rebounds available to be had.  A simple adjustment would be to use offensive and defensive rebound percentage - that is the percent of available offensive and defensive rebounds a team collects.
\begin{itemize}
\item Strength of Schedule
\item Free throw percentage
\item Field goal percentage
\item effective field goal percentage
\item rebounding percentage
\item scoring percentage
\end{itemize}
\andyc{How to best present this section?  Maybe use it as an opportunity to talk about some modern stats(tempo adjusted): e.g. rebound percentage vs. rebounds}
\subsection{Common Ratings Components} Another data source is the huge number of rating systems available on the web, which use the factors mentioned in the previous section in various ways.  With the large number of people working in this area, Ken Massey's website www.masseyratings.com contains nearly 70 different systems, it is no surprise that these rankings work quite well and we have found it difficult to make major improvements over these ranking systems.  Otherwise, we might use the arbitrage situation for financial gain rather than publish this paper.  While many of the algorithms are proprietary, we provide an overview of the main points for select notable algorithms.  Some aspects of these equations are proprietary so complete details are not possible in each case. 
\subsubsection{RPI}
The most well known rating system may very well be the Ratings Percentage Index (RPI).  The RPI was created in 1981 as a tool for evaluating teams for admission and seeding to the NCAA basketball tournament.  The RPI calculation uses three components, winning percentage (WP), opponents winning percentage (OWP), and opponents opponents winning percentage (OOWP), with weights specified as:
\begin{eqnarray*}
RPI = WP *.25 + OWP * .50 + OOWP *.25.
\end{eqnarray*}
The RPI is often criticized for heavy reliance on winning percentage and failure to account for other indicators such as point differential that illuminate the team strength.

\subsubsection{Sagarin}
Another well known ratings system is Jeff Sagarin's computer ranking system, known simply as the Sagarin rankings.  Sagarin rankings are a staple, because of both the longevity  - they have been used since 1985 - and the quality.  The actual methodology is unknown, but the Sagarin rating is actually a composition of three separate models.  One model uses only wins or losses without regard to point differential, while the other two focus primarily on point spreads.  The Sagarin rankings are unique in that difference in the team ratings represents the expected point spread for that matchup on a neutral court.  There is also a home court advantage of 3.38 points built into Sagarin's system.

\subsubsection{Pomeroy}
The Pomeroy rankings are issued by Ken Pomeroy and largely driven by the Pythagorean Expectation a formula developed by Bill James for baseball prediction.  The Pythogorean expectation is
\begin{eqnarray}
E[P(Win)] = \frac{\text{points scored}^c}{\text{points scored}^c + \text{points allowed}^c},
\end{eqnarray}
where $c$ is a constant, Pomeroy uses 10.25.  Rather than actual points scored Pomeroy uses adjusted and defensive efficiencies as inputs to the Pythogorean expectation.  Pomeroy cites that this method gives an average error of 8.25 based on backtesting and states
\begin{quotation}\noindent I don�t think you can�t come up with a prediction method that will have an error of less than eight points. And if you can, don�t tell anyone! Because that would be a really good system. That should also tell you a lot about why it�s difficult to anticipate what will happen in a single contest between teams. It�s also a good illustration of the large role randomness in any single game. So even if you know it all, you can�t possibly know it ALL.
\end{quotation}
This quote illustrates the difficulties inherent in basketball prediction.  \andyc{move quote elsewhere?}
\subsubsection{Logistic Regression - Monte Carlo}
The logistic regression Monte Carlo (LRMC) method detailed in \cite{Kvam2006}, \cite{mark2010} is a two step procedure used to produce ordinal rankings of each team.  The first step evaluates every game played during the season to compute the probability that the winning team is better than the losing factor.  This step uses the score at the end of regulation (e.g. overtime results are not included) and the home team in a logistic regression setting.  The second step uses these probabilities in a Markov chain to produce the ordinal rankings.  Specifically, each team is given a state in the chain based on the head-to-head probabilities and the ordinal ranking is a result of the ordering for the steady state probabilities of the chain.
\subsubsection{Rating Comparison}
Table \ref{tab:ranks} contains pre-tournament rankings from common sources.  The table includes any team ranked in the top-10 by either of the four previously discussed rankings systems as well as the eventual national champion Connecticut, who didn't make any of these top 10 lists.
\begin{table}[ht]
\caption{Pre-tournament Ranking Comparison}
%\footnotesize
\centering
\begin{tabular}{c|cccc|c}
  \hline
  \hline
 Team & Sagarin Rank &  Pomeroy Rank & RPI & LRMC Rank & Ave. Rank  \\ 
  \hline
 Arizona         & 1  &1    & 2     & 2 & 1.5  \\
 Florida          & 3  &3    &1      &3 & 2.5\\
  Kansas         & 6  &8    & 3    & 4& 5.25\\
 Louisville      & 2  &2    & 19   & 1 & 6\\
 Villanova      & 4  &7    & 5    & 9 & 6.25\\
 Virginia         & 5  &4     &8    &8 & 6.25\\
 Wichita St    & 12 &5      & 4    &5 & 6.5\\
 Duke             & 7  &6     &9     &6 &7\\
  Creighton &  11     &   9       & 10     &7 & 9.25\\ 
 Wisconsin  &   9   &13   & 6   &  11 & 9.75\\
 Michigan St & 8  &10    & 18 & 12& 12\\
 Michigan & 10 & 15& 11& 16& 13\\
 UCLA & 15& 18& 14&10 &14.25\\
 Iowa St &13 &23  &7 &19 &15.5 \\
  \hline
  Connecticut & 24& 26& 22&26 &24.5\\
  \hline
   \hline
\end{tabular}
\label{tab:ranks}
\end{table}

There are some similarities but each of the ranking system has its own flavor.
\section{Decision Theory - Optimal Strategies}
\subsection{Probabilistic Forecasting}
\subsection{proper scoring rules}
\subsection{Strategy: Min Risk or Max Expected Earnings?}
\andyc{ The entire process could be simulated using the historical probs of seed X vs. seed Y, this would give some range on the ideal (oracle) solution in which the probs are known}

%\section{Popular Methods}  This section provides a review of several existing methods for tournament prediction.
\subsection{Rating Based Methods} 
The seeds and ratings discussed earlier contain implicit or explicit means for tournament prediction.  For instance the Sagarin rankings are designed to reflect the point spread between two teams, incorporating a term for the variance of the outcomes under a Bayesian framework provides an efficient way to compute winning probabilities.  Similarly for any of the other rankings coefficients for predicting point spreads or in a binary regression can be computed. A seed based probability is shown in \cite{schwertman1996}.  A more sophisticated approach using point spreads along with Sagaring rankings is shown in \cite{carlin1996}.
\subsection{Ensemble Methods}
Combining the ratings from several different sites is a popular strategy.  In fact this is the technique used by Nate Silver's 538 methodology.  In particular Nate Silver's model incorporates ... \andyc{cite here}
\subsection{Wisdom of the Crowds?}
Another popular option would be to use the \emph{Wisdom of the Crowds}.  ESPN provides a mean probability across every bracket submitted to the site.
\scotlandc{include all into sensitivity study with Lucas}

\subsection{Evaluation}
\andyc{Xinran's piece here}


\section{Modeling}
This section details common prediction methods and then highlights our novel methodology that captures matchup specific factors. As a means for motivation, consider claims of the type ``\emph{team i is a tough matchup for team j due to their ...}'' often made by sports broadcasters. There are two ways to consider this statement: (i) the overall team strength of team $i$ will be problematic for team $j$ or (ii) team $i$ has certain tendencies above and beyond their team strength that will pose difficulties for team $j$. We outline a general framework for the first case, specifying models that account for differences in team strength. However, for the second case a different approach is needed to analytically quantify characteristics that pose difficulties for a given team. Many traditional methods, particularly those based on rankings, impose a characteristic we deem transitivity on predictions. While transitivity is fully in coming sections, the key point is that models with this property rely on estimates of team strength and are unable to adapt and computer specific tendencies of a given matchup. Hence, we introduce the Nearest-Neighbor Matchup Effect which captures characteristics of specific match ups and doesn't adhere to transitivity in predictions.  

\subsection{Data Treatment}
For our purposes game summary data is used, but another approach not explored here focuses on simulating each sequence in a game as in \cite{vstrumbelj2012} rather than the final outcome. There are a few necessary data considerations when modeling game outcomes, specifically whether the outcome of a game is binary (win/loss) or continuous (point differential) as well as whether linear or non-linear modeling (in the predictors) should be used. For the first dilemma involving whether the outcome should be modeled in a binary or continuous sense, point spread provides a means for eliciting the relative strength of one team. Although as any basketball fan can attest to the final score is often not indicative of how the closeness of the game. Nevertheless, point spreads are informative about the relative strength of one team compared to another. In practice, given a posterior distribution of point spreads, the transformation to win probability is straightforward.  In regards to the second consideration, our experience showed that non-linear methods such as CART provided little extra predictive power when compared to linear methods. This is particularly the case when using a selection of rating systems.

\subsection{Relative Strength Models}
Adopting the continuous treatment of game outcomes (e.g. point spread) we next detail models designed to capture the relative strength of two teams. The general form for the relative strength models is:
\begin{eqnarray}
Y_{ijk} = g_1(X_{ij}) + g_2(X_i) + g_3(X_j) +  \epsilon_{ijk}
\label{eq:RS}
\end{eqnarray}
where $Y_{ijk}$ is the point differential between teams $i$ and $j$ for matchup $k$.  The covariate matrix $X_{ij} = \{x_{ij1}, ..., x_{ijp_1}\}$ corresponds to the difference in covariates for teams $i$ and $j$, where for instance $x_{ij1} =$ Sagarin rating team $i$ - Sagarin rating team $j.$  The covariate matrix $X_l = \{x_{l1}, ..., x_{lp_2}\}$ contains predictors for team $l$, and $\epsilon_{ijk} \sim N(0,\sigma^2)$ is the error for the $kth$ matchup between teams $i$ and $j$. A simple example,but surprisingly useful example of Equation~\ref{eq:RS} would be a simple linear model:
\begin{eqnarray}
Y_{ijk} = X_{ij}\beta + \epsilon_{ijk}.
\label{eq:RS_Linear}
\end{eqnarray}
\subsection{Calibrating Probabilities}
Given a posterior predictive distribution for point spread between teams $i$ and $j$, $Y_{ijk},$ it is a straightforward transformation to recover win probability. Consider Figure \ref{fig:winprob} which displays the distribution of point differential and corresponding win probability for a team favored to win by 5 points (e.g. $E[Y_{ijk}]=5$).
\begin{figure}[h!]
\centering
\includegraphics[width=.9\textwidth]{WinProb.pdf}
\caption{Calculating win probability from point differential}
\label{fig:winprob}
\end{figure} 
While the actual variance of the posterior predictive distribution will vary, the distribution displayed in Figure \ref{fig:winprob} is $Normal(5,11^2).$

\subsection{Popular Methods}  This section provides a review of several existing methods for tournament prediction.
\subsubsection{Rating Based Methods} 
The seeds and ratings discussed earlier contain implicit or explicit means for tournament prediction. For instance the Sagarin rankings are designed to reflect the point spread between two teams, incorporating a term for the variance of the outcomes under a Bayesian framework provides an efficient way to compute win probabilities. Similarly for any of the other rankings coefficients for predicting point spreads or in a binary regression can be computed. A seed based probability is shown in \cite{schwertman1996}. A more sophisticated approach using point spreads along with Sagarin rankings is shown in \cite{carlin1996}. A method that allows team strength to vary in shown in \cite{glickman1998}.
\subsubsection{Ensemble Methods}
Combining the ratings from several different sites is a popular strategy. In fact this is the technique used by Nate Silver's 538 methodology. In particular Nate Silver's model incorporates 7 sets of rankings: Sagarin's, Pomeroy's, LRMC, Sonny Moore power ratings, ESPN's Basketball Power Index (BPI), NCAA selection committee ``S-curve'', and Associated Press preseason poll along with injuries and distance required to travel to construct power ratings (\cite{silver}). Similar to Sagarin's ratings, the rating difference between two teams are designed to reflect an estimated point spread. In general model averaging proved useful, smoothing out over fitting or model biases.  \cite{silver} goes on to say:
\begin{quote}
One of the ways I was able to look smart over the past six years, during which time I spent a lot of effort on political forecasting, was by betting on the favorite. I wasn't literally placing bets, mind you (unless you want to count my proposed bet with Joe Scarborough). But for some reason, in political prognostication, you can be regarded as a savant just by pointing out that the favorite is probably going to win.

The standard in sports prediction is higher. And this year's NCAA basketball tournament is designed to make me look dumb. There aren't any favorites. Sure, some teams are better bets than others. (I wouldn't advise staking your fortune on Cal Poly.) But the team that our statistical model regards as the favorite to win it all, Louisville (more on the Cardinals in a moment), has just a 15 percent chance of doing so. In other words, there�s an 85 percent chance that Louisville won't cut down the nets again and that I'll be wrong.
\end{quote}
\andyc{Consider sliding this up to intro along with Pomeroy quote}
\subsection{Nearest-Neighbor Matchup Effects}
The nearest-neighbor matchup effects are tailored for the second scenario in which there exist team level characteristics - above and beyond team strength - that contribute to win probability. For instance, maybe a certain team struggles with taller teams that rebound well. When facing an opponent with these attributes the team would expect to perform worse than the difference in team strengths would suggest. This is plausible as the team strength is calculated across a variety of opponents, hence we search for a subclass of opponents in which the strengths differ from the overall mean. Our procedure is a three step process: (i) fit a relative strength model of the form specified in Equation~\ref{eq:RS}, (ii) identify neighbors, by finding similarities between the current opponent and past opponents for each team, and (iii) calibrate the matchup adjustment. Fitting of the relative strength model follows the same form as previously described and will not be rehashed in this segment. Our example model which is fully detailed in the following following section follows the form of Equation \ref{eq:RS_Linear} using the Sagarin ratings as predictors.
\subsubsection{Choosing Neighbors}
When choosing neighbors for a matchup between team $i$ and team $j$, we need to identify past opponents of team $i$ with similar attributes to team $j$ and past opponents of team $j$ with similar attributes to team $i.$ The idea is to identify how the performance changes against that subset of opponents relative to the overall relative strength computed on the entire set of opponents a team as faced.

There are a multitude of ways to select the neighbors. In particular one needs to consider what variables to include for selecting neighbors, how should those variables be weighted if at all, and how many neighbors should be selected. We consider a large set of team level data from which a 5-nearest neighbor approach is calculated. In particular we use over twenty variables equally weighted for the nearest neighbor calculation including: team height, adjust tempo, percentage of scoring from 3 pointers, offensive rebound percentage, free throw percentage, block percentage, steal rate, and many others.

Typically this procedure would be done analytically, however, user input can also be solicited. For instance, suppose that a user decided Mercer was similar to Wake Forest and Clemson. In this case current and ongoing work in Bayesian Visual Analytics (BAVA) framework detailed in \cite{house2010}  and \cite{hu2013} provides a principled routine for visualizing teams and taking user input of similarities to create a method for computing distance between teams. Specifically this is a way to weight clustering variables to reflect user preferences.
\subsubsection{Matchup Adjustment}
The idea of the matchup adjustment is to quantify how much a team underperformed (or over performed) relative the expected team strength for a subset of teams similar to the current opponent. For instance, if team $i$ was two points better than expected against teams similar to $j$, then it would be reasonable to assume that team $i$ would perform better against team $j$ as well.  Assume a simple linear model from Equation \ref{eq:RS_Linear}, then the predictive distribution for a matchup between team $i$ and team $j$ now becomes 
\begin{eqnarray}
p(Y_{ij}|X_{ij}, \beta,\sigma^2,\mathcal{N}_i^k(j),\mathcal{N}_j^k(i), \rho) &\sim& N(\epsilon_{ij}, \sigma^2) \label{eq:ME}
\\
\epsilon_{ij} &=& X_{ij} \beta + \rho(\mathcal{N}_i^k(j) -\mathcal{N}_j^k(i)),
\end{eqnarray}
where $\mathcal{N}_j^k(i)$ is the average residual for the team $i's$ $k$ past opponents most similar to team $j$ and $\rho$ is a tuning parameter $\in [0,1]$ that controls the amount of information passed from similar neighbors. The same idea applies to more general model specified in Equation \ref{eq:RS}.

\subsubsection{Calibrating the Matchup Adjustment}
The natural support of $\rho$ would be between zero and one.  The interpretation of the extreme points is rather intuitive - with $\rho = 0$ Equation~\ref{eq:ME} reverts to Equation~\ref{eq:RS} and with $\rho = 1$ the entire residual for similar teams is retained. To select $\rho$ in practice, we recommend a historical analysis to calibrate $\rho$ for the current year's predictions. For the relative strength model and nearest neighbor variables used in this work, $\rho$ near 0.2 performed optimally across the historical data.   \andyc{maybe add the plot Marcos suggested here}.

\subsection{A note about Transitivity}
The transitive property states if $A>B$ and $B>C$ then $A>C$. In terms of basketball prediction consider:
\begin{eqnarray}
P_{A,B} > 0.5 \quad \& \quad P_{B,C} > 0.5 \rightarrow P_{A,C} > 0.5,
\label{eq:trans}
\end{eqnarray}
where $P_{I,J}$ is the probability of team I defeating team J.  We deem Equation \ref{eq:trans} a transitive property for basketball prediction. That is if team A is expected to beat team B and team B is expected to beat team C, then team A should also defeat team C. Note these are probabilities not true outcomes, due to the parity in basketball inferior teams can and often do defeat stronger teams. Any sort of relative strength model would require this transitive ordering holds, home court effects non-withstanding. On the other hand, our matchup effects modeling approach can determine if the strengths of a given team present difficulties for a specific team meaning the transitive property is not required to hold and $P_{A,B} > 0.5 \quad \& \quad P_{B,C} > 0.5\quad \& \quad P_{A,C} < 0.5$ is valid.

%\section{Model: Nearest-Neighbor Matchup Effects}
Sports broadcasters often make claims similar to \emph{Team X is a tough matchup for team Y due to their ... }.  There are two ways to consider this statement: (1) the overall team strength of Team X will be problematic for Team Y or (2) Team X has certain tendencies above and beyond their team strength that will pose difficulties for Team Y.  For the first case, models specified by Equation \ref{eq:generic} will account for differences in team strength.  However, for the second case a different approach is needed to analytically quantify characteristics that pose difficulties for a given team.  We introduce the Nearest-Neighbor Matchup Effect which captures characteristics of specific match ups.  For instance in hindsight, a glance inside the crystal ball would have revealed that Duke might struggle against Mercer.  It is not clear if this was the ten percent of instances that a 14 seed defeats a 3 seed, \andyc{check this historical frequency} or whether an uncharacteristically difficult matchup to the Mercer's characteristics for third seeded Duke.  Perhaps an astute observer - or clustering algorithm - would have recognized that Mercer was a team with similar characteristics to Clemson and Wake Forest, two teams Duke struggled against during the regular season. Then the matchup effect would have revised to initial probability of Duke winning to account for the fact that Mercer has similar characteristics to Clemson and Wake Forest resulting in a winning probability of 72 percent compared to the initial 90 percent estimate.  Computing matchup effects is a three step procedure: (1)  the typical model as in Equation~\ref{eq:generic} is fit, (2) for each matchup, past opponents most similar to the current matchup are identified, and (3) an adjustment is introduced that accounts for past performance against similar teams. 
\subsection{Relative Strength Models}
The general form for the relative strength models follows below:
\begin{eqnarray}
Y_{ij} = X_{ij} \beta + \epsilon
\label{eq:ME}
\end{eqnarray}
where $X_{ij}$ coresponds to the difference in predictors for teams $i$ and $j.$  Most commonly $X_{ij}$ will consist of differences in rankings and seeds between the two teams. 

\subsection{Choosing Neighbors}
There are a multitude of ways to select the neighbors.  In particular one needs to consider what variables to consider for selecting neighbors, how should those variables be weighted if at all, and how many neighbors should be selected.  We consider a large set of team and player level data from which a k-nearest neighbor approach is calculated.

Typically this procedure would be done analytically, however, user input can also be solicited.  For instance, suppose as in the earlier example that a user decided Mercer was similar to Wake Forest and Clemson.  In this case, Bayesian Visual Analytics (BAVA) provides a principled routine for visualizing teams and specify similarities.
\subsection{Matchup Adjustment}
The idea of the matchup adjustment is to quantify how much a team underperformed (or over performed) relative the expected level for a subset of teams similar to the current opponent.  So if team $i$ was two points better than expected against teams similar to $j$, then it would be reasonable to assume that team $i$ would perform better against team $j$ as well.  Then the predictive distribution for a matchup between team $i$ and team $j$ now becomes
\begin{eqnarray}
p(Y_{ij}|X_{ij}, \beta,\sigma^2,\mathcal{N}_i^k(j),\mathcal{N}_j^k(i), \rho) \sim N(X_{ij} \beta + \rho(\mathcal{N}_i^k(j) -\mathcal{N}_j^k(i)), \sigma^2),
\label{eq:ME}
\end{eqnarray}
where $\mathcal{N}_j^k(i)$ is the average residual for the team $i's$ $k$ past opponents most similar to team $j$ and $\rho$ is a tuning parameter $\in [0,1]$ that controls the amount of information passed from similar neighbors. 
\subsection{Tuning $\rho$}
The natural support of $\rho$ would be between zero and one.  The interpretation of the extreme points is rather intuitive - with $\rho = 0$ Equation~\ref{eq:ME} reverts to Equation~\ref{eq:generic} and with $\rho = 1$ the entire residual for similar teams is retained.

\section{Evaluation}
\subsection{Popular Methods}
\andyc{Xinran's piece here}
March Madness prediction also drew attentions of many sport analysts and medias. We selected three popular predictions to examine their performances. These three models are: Nate Silver's prediction from 538.com, Ken Pomeroy's from kenpom.com and the Power Rating model from ESPN Insider. We will apply the Kaggle loss function to score these three models. As Nate Silver and Ken Pomeroy only published probabilities of each team advancing to a certain round, our comparison only considered the first round games (round of 64). It is also worth mentioning that the probabilities of the four teams who won their first-four games winning the round of 64 is computed as \(a/(a+b)\) where \(a\) and \(b\) are the probabilities of the team and its opponent wins. 

The first thing to notice is that all three models concurred in all win-loss predictions of except one (Gonzaga vs OKST, 8 vs 9 in West). All three models have a similar pattern of confidence. By confidence, we meant how far away predictions are from 0.5, e.g., predictions of 0.95 and 0.48 have confidence of 0.45 and 0.02, respectively. In all three models, most predictions either have high confidence (close to 0.5) or low confidence (close to 0 ), while moderate confidence are relatively rare. The reason may be a strategical adjustment for easy games. This idea may also be useful in future Kaggle competitions as to score as much as possible for easy games.

In terms of level of confidence, Silver adapted a more aggressive style than the other two. His predictions, on average, have an 0.2818 confidence while Pomeroy averages the lowest, 0.2388. However, it is showed that confidence doesn't affect the result much. For the first around games, Pomeroy has the best score of 0.4632, Silver scored 0.4664 and 0.4709 for ESPN. 

We have also examined the luck factor with these popular models. After flipping the results of five games that went overtime in round of 64, we noticed a completely reverse of the rankings of scores. The ESPN model now ranks first with 0.4631 and Pomeroy ranks third with 0.5027, Silver remains second with 0.5025. This result supported our previous conclusion that luck is a substantial factor in Kaggle. 

A further comparison showed us that the two analyst predictions are quite similar and both outperforms the ESPN model. This may be because ESPN model is strictly transitivity. Human analysts' capability of capturing intangible information regards the teams, such as  morale or playing style, could have played a vital role in avoiding transitivity and obtaining more reliable predictions.         


\subsection{Nearest Neighbor Matchup Effects}
To demonstrate the efficacy of our method, we first fit Equation~\ref{eq:RS} using a well known rating system, the Sagarin ratings.  Then, $\rho$ is calibrated based on historical results from the previous seven years NCAA tournaments.  Note seven years was chosen as this is the complete history of the team and player level characterstics used to find neighbors of teams.   The log loss for 2014 for the entire range of $\rho$ can be seen in Figure~\ref{fig:result} and the value that was selected based on historical results $0.2$.
\begin{figure}[h!]
\centering
\includegraphics[width=.7\textwidth]{results_2014.png}
\caption{Log loss for no matchup effect = red, log loss for optimized $\rho$ = blue}
\label{fig:result}
\end{figure}

The selected $\rho$ value results in a reduction in loss.  A modest gain is also seen in classification error from (0.365 to 0.350) although this is only a single game difference.  The matchup effect, particularly with smaller $\rho$ values, will have a lesser effect on classification error than that off the loss functions like the log loss.  This is because it will only shift the expected point differential a fairly small margin, so the only games in which classification error would change are those that are nearly dead heat games to begin with.

To illustrate the matchup effects, consider Table~\ref{tab:change} which contains the ten games that saw the largest shift in expected point differential.  This table contains expected point differentials (team1 - team2) and probabilities of team 1 winning under a standard relative strength model using the Sagarin ratings as well as the adjusted results using the Nearest Neighbor Matchup Effects.  The table also contains the realized loss for the each resulting game.
\begin{table}[h!]
\caption{Ten games with largest point differential change}
\footnotesize
\centering
\begin{tabular}{|cc | ccc | ccc | c|}
  \hline
  \hline
 team 1 & team 2 & Point Diff & Prob & Loss & Point Diff:ME & Prob:ME & Loss:ME & winning team \\ 
  \hline
 Cal Poly & Wichita St & -18.69 & 0.04 & 0.04 & -17.10 & 0.06 & 0.06 & Wichita St \\ 
 Connecticut & St. Joseph's &4.29 & 0.65 & 0.43 & 6.18 & 0.71 & 0.34 & Connecticut \\ 
 Dayton & Stanford & -2.16 & 0.42 & 0.86 & 0.94 & 0.53 & 0.63 & Dayton \\ 
 Dayton & Syracuse & -6.34 & 0.28 & 1.27 & -4.05 & 0.36 & 1.03 & Dayton \\ 
 Kentucky & Michigan & -3.71 & 0.37 & 1.00 & -2.08 & 0.42 & 0.86 & Kentucky \\ 
 UMass & Tennessee &-3.05 & 0.39 & 0.49 & -4.83 & 0.33 & 0.40 & Tennessee \\ 
 Memphis & Virginia & -6.34 & 0.28 & 0.33 & -8.91 & 0.21 & 0.23 & Virginia  \\ 
 Michigan & Tennessee & 5.37 & 0.69 & 0.37 & 3.49 & 0.62 & 0.47 & Michigan\\ 
 Michigan & Texas & 8.05 & 0.77 & 0.26 & 5.85 & 0.70 & 0.35 & Michigan \\ 
 Syracuse & W. Michigan & 12.65 & 0.88 & 0.13 & 15.01 & 0.92 & 0.09 & Syracuse \\ 
   \hline
   \hline
\end{tabular}
\label{tab:change}
\end{table}
On this particular subset of games, the matchup effects model performs considerably better than the typical model under the log loss (.446 to .520).  As the other games see minimal matchup effects, the results are essentially the same.  
\section{Luck}
how sensitive is the leader board. It's natural to think 2nd place almost won, but how close was 20th place to winning. (Lucas: sensitivity study)

%\section{Inference on Transitivity}
\andyc{This may get cut}
The transitive property states if $A>B$ and $B>C$ then $A>C$.   In terms of basketball consider:
\begin{eqnarray}
P_{A,B} > 0.5 \quad \& \quad P_{B,C} > 0.5 \rightarrow P_{A,C} > 0.5
\label{eq:trans}
\end{eqnarray}
, where $P_{I,J}$ is the probability of team I defeating team J.  Then Equation \ref{eq:trans} can be considered a transitive property on basketball match ups.  That is if team A is expected to beat team B and team B is expected to beat team C, then team A should also defeat team C.  Any sort of rank based approaches would assume this transitive ordering, home court effects non-withstanding.  Note these are probabilities not true outcomes, due to the parity in basketball inferior teams can and often do defeat stronger teams.  Nevertheless, our modeling approach can determine if the strengths of a given team present difficulties for a specific team resulting in the transitive property not necessarily holding.

\section{Recommendations}
things that work and don't work. How do you train the models� for each season, do you train the model to predict that years tournament. Or, do you use all 5 years worth of data to predict a single year? The first choice is obviously better. "The data is probably more important than the models- cite from the kaggle winners"

-subsection in recommendations: How should a typical user use this to figure out their bracket?
 - one strategy is take the most probable. is there another? (Marcos)

\section{Discussion}
other data such as injury reports (Nate uses this)

Discussion: Is the score at 2 minutes to go better than the final score? In the last minutes of the game, wonky stuff happens

An untapped section would be looking at player level data and potentially player level match ups. 

Team strength is most definitely not constant over the season, state space framework to model the evolution?
%%%%%%%%%%%%%%%%%%%%%%%%%%%%%%%%%%%%%%%%%%%
%%%%%%%%%%%%%%%%%%%%%%%%%%%%%%%%%%%%%%%%%%%
%%%%%%%%%%%%%%%%%%%%%%%%%%%%%%%%%%%%%%%%%%%

\bibliographystyle{DeGruyter}
\bibliography{refsJQAS}

%%%%%%%%%%%%%%%%%%%%%%%%%%%%%%%%%%%%%%%%%%%
%%%%%%%%%%%%%%%%%%%%%%%%%%%%%%%%%%%%%%%%%%%

%%%%%%%%%%%%%%%%%%%%%%%%%%%%%%%%%%%%%%%%%%%


%%%%%%%%%%%%%%%%%%%%%%%%%%%%%%%%%%%%%%%%%%%

\end{document}
