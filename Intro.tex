\section{Introduction}
 Every March, millions of people take to an American tradition, filling out an NCAA tournament bracket.  Typical strategies include listening to so-called experts or following one's intuition, although the rise in sports analytics and the popularity of sites such as Nate Silver's fivethirtyeight.com are increasing the use of data-driven methods.  With this work we review analytic approaches to forecasting NCAA tournament games and introduce our modeling framework designed to capture matchup effects.   Many common methods for predicting outcomes assess overall team strength using ranking based metrics.  In addition to quantify overall team strength, our methodology provides a data-driven approach to assess team by team match ups and address statements of the form: \emph{"Team Y is a tough draw for team X due to their tempo, size, athleticism, three point shooting, ect.."}.  The essence of of the matchup effects is to discern the existence and magnitude of team-by-team match ups.  While there certainly is a high degree of uncertainty involved in predicting outcomes and a small number of games for evaluation, we demonstrate the efficacy of our matchup effects.  We predicted Duke would lose.  THE END
\subsection{Common Prediction Methods}
Many commonly used methods or models include a seeds based approach or using one of several rankings systems: Sagarin, Pomeroy, ESPN BPI, ect..
\subsection{Other}We should be explicit about the Kaggle comp. vs. just filling out a bracket. For kaggle, there are no broken brackets.
\subsection{LitReview}
Literature Review: What people have done for predicting tournaments before (both bball and other sports are fine), March madness in general (how many people watch, how much revenue is it worth). Predicting human performance in sports (it's hard), history of Kaggle.
