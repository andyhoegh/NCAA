\section{Introduction}
 Every March, millions of people take to an American tradition, filling out an NCAA tournament bracket.  While there are a multitude of ways to fill out a bracket,typical strategies include using the tournament seeds, listening to so-called experts on tv or the internet, or following one's intuition.  Recently, the rise in sports analytics and the popularity of sites such as Nate Silver's fivethirtyeight.com are increasing the visibility of data-driven methods for NCAA tournament predictions.  Additionally, unique tournament challenges such as the one hosted by Kaggle in 2014 call for winning probabilities for each potential matchup essentially requiring computational methods.  This work contains three major components: (i) a review analytic approaches to forecasting NCAA tournament games, (ii) an introduction our modeling framework designed to capture matchup effects, and (iii) a discussion the large effect of chance plays on tournament outcomes and hence, modeling competition results.
   
\subsection{March Madness}  Paragraph or section on March madness in general (how many people watch, how much revenue is generated, ect...)
\indent The college basketball tournament in March in the US was suggested by the coach Harold Olsen from Ohio State University and was organized by The National Collegiate Athletic Association (NCAA) during 1939. "March Madness" has been used to refer the college basketball tournament especially and this term is owned by NCAA as a registered trade mark. This term was firstly used by H.V. Poter to describe the NCAA Basketball Madness tournament and published his article, "March Madness" in 1939. As the NCAA basketball tournament gets popular, filling the bracket which teams will win becomes the traditional event in March for American basketball fans. In addition, since 2011, Turner Sports and CBS Sports have broadcasted every tournament game nationally. Due to a nationwide broadcast, an advertisement revenue each year is getting increasing and eventually March Madness in 2013 produced an advertisement revenue,$\$ 1,152$ million, the highest recorded ever(Source:Kantar Media).\\
%http://kantarmedia.us/press/march-madness-generated-1-billion-ad-revenue-2013
'
Energized the popularity of March Madness, Kaggle, which founded in 2010 by Anthony Goldbloom, announced a competition, "March Madness Learning Mania" to build the predictive model for NCAA college basketball tournament.

\subsection{Overview of Kaggle} History, ect...


\subsection{Tournament Prediction}


Methods for predicting tournaments, (ball and other sports), difficulties of prediction human performance in sports (its hard)
